Graph databases are an alternative to traditional Relational Database Systems
(RDBMS), which improve access speeds for some queries by prioritising access to
related entities, rather than expecting traversal through a fixed schema. This
design can more naturally represent relationships between data in many fields,
from the analysis of gene networks in biology to social network recommendation
engines.  It can provide huge performance improvements for graph-centric
queries, such as shortest path queries. 

For other queries, however -- such as aggregation of otherwise unrelated data
items -- it is far more efficient to rely on a known schema over a fixed set of
rows for fast access. There are many trade-offs to be made between the RDMBS and
Graph Database models, but the core division is that where a graph database prioritises access to topological information, an RDBMS equally prioritises attributes.
In practice, however, although it is certainly the case that some queries benefit more than others from this prioritisation, it is rarely the case today for a business' operations to rely on a single type of query. 

Road networks are often used as an important example of a highly connected
dataset. However the reality is that, although relationships between junctions
can be well modelled as a graph, this is not the most efficient data format
for all queries.  Calculating the shortest path between two vertices, for
example, will often perform faster using a graph database, since much of the
graph can be ignored during the search, where a relational database must
consider each edge equally. On the other hand, performing queries which
concern all junctions equally -- such as finding some point of interest in the
network -- is much more efficient using a relational database with a fixed
schema, since computation can rapidly jump through memory using known
intervals to retrieve desired attributes. This heterogeneity of possible query
workloads, coupled with the common uncertainty at design-time about which part
of an application will be most frequently used, makes it difficult to choose
an ideal database system for any particular application.

I will return to this example of a road network throughout this paper, as it
provides helpful explanation for the projects' motivation. A company building a navigation
application, for example, may be wary of relational databases due to the risk of 
of high execution times for topology-focussed queries such as pathfinding. 
Users also need to be able to search for points of interest in the map. Since the searches
may contain errors, the company cannot simply use the search terms as indexes for fast
information retrieval, but must instead perform a string comparison against every point individually. For this kind
of attribute-focussed workload, a graph database would perform poorly. The most common kind of user query,
however, is likely to be of the form ``What is the shortest path to get to `Emmanuel College'?'' This 
query has both attribute-focussed and topology-focussed components, which will run into problems
using either of the approaches currently commercially available. Grapht aims to
provide a middle ground between the two which avoids these bottlenecks.


\section{Research Goals}

The optimisations made to increase the performance of graph databases for
certain problem necessarily sacrifice performance for others. Thus it seems
optimistic to aim to create a system which performs both as well as production
graph databases for topology-focussed problems \textit{and} as well as
PostgreSQL for attribute-focussed problems. Instead, the aims for this project
were to primarily improve on the worst-case performance of these systems, to
provide a more homogeneous performance profile across both types of query. By
doing this, we can provide a system which provides a higher average-case
performance in situations where graph-centric queries and row-centric queries
are both used. More concretely, there were two main goals for this project.
Note that for this research project, we limit our attention to data-retrieval
queries only.


% Relational database engines perform poorly for graph-centric queries such as path-finding.
% A hybrid engine should perform better than this, even if it cannot achieve performance on
% par with purely graph-centric engines.


% Conversely, graph engines perform poorly for row-centric queries such as aggregation
% over all nodes, and a hybrid engine should aim to perform better than this.

\subsection*{1: Increase performance of hybrid queries compared to both relational and graph databases}

Some queries have multiple components to them, some of which would best be
handled by a graph engine, while others best by a relational system. As
discussed previously, we may wish to identify some junction by name, and find
a path to it through a road network. Under either a relational or graph
database, some component of the query would necessarily perform badly and
cause a performance bottleneck. A hybrid system ought to improve on these
worst-case performances and not be affected so badly by either bottleneck. In
so doing, it may thus outperform both systems for this combination queries.

\subsection*{2: Expose a coherent interface for this hybrid system}

Relational databases have a well-established query interface in the form of
\textit{SQL}.  Queries in SQL aim to filter out particular rows from a global set, rather than limiting attention to a connected subset of entities, making it difficult for a
programmer to even express graph-centric queries. Although the graph database
community have not yet settled on a single query language to act as SQL's
graph-centric counterpart, the two most popular languages -- Gremlin and Cypher
-- suffer from the opposite problem: they are not well suited to expressing
row-centric queries which consider all vertices in the graph. Since a hybrid system
must aim to be equally well suited in terms of performance to either sort of
query, it should not give precedence to either in terms of interface. Graph
queries should be as easy to express using the hybrid system as row-centric
ones.


In the next chapter, I will first give a brief introduction to the
technologies involved in both relational and graph databases, including their
relative strengths and weaknesses. I will continue by providing a brief
overview of the surrounding research landscape, particularly  focussing on
previous attempts to bring together graph and relational systems. Chapter
\ref{ch:implementation} discusses the implementation of a research prototype
aiming to satisfy both of my research goals: \textit{Grapht}. The success of
this prototype is experimentally evaluated in Chapter \ref{ch:evaluation},
along with direct comparisons between Neo4J and PostgreSQL. These comparisons
serve not only to provide a baseline against which to measure Grapht's
performance, but also to validate the claims made above about the relative
performance of the two systems in satisfying different types of query.
Finally, Chapter \ref{ch:conclusion} summarises the findings made, and
discusses possible directions for future research.
