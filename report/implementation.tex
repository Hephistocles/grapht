
In aiming to produce a hybrid system somehow equally well suited to both
relational and graph-centric problems, two approaches are possible. The first would be to create a
novel system from scratch which fully commits to neither underlying storage
strategy; following instead some middle path to achieve optimality. A second
strategy -- and the one employed here in the design of Grapht -- is to extend
the capabilities of one system to mitigate its weaknesses by taking inspiration
from the other system. In particular, Grapht extends a durable relational data
store with a graph-centric in-memory store.

In this chapter, I first give a brief introduction to the different components
involved in resolving a query through Grapht. Following that, I describe the
two main components in more detail: the graph store prefetcher and the query
processor.






\section{System Overview}

As mentioned earlier, Grapht extends an RDBMS -- a graph query
layer is ``grafted'' onto an underlying RDBMS. The underlying RDBMS should
contain two relations in particular: one representing vertices in a graph,
and one representing edges, with endpoints represented by keys to the vertex
relation. This data layout is already a common way to describe a graph in a
relational environment, with self-\texttt{JOIN} queries on the edge relation
used to explore paths consisting of multiple hops. In this way, Grapht often
requires no change to the data format, and can simply be added to a relational
database with very little work. Internally, Grapht is made up of two
parts: a row-centric and a graph-centric handler. Both components communicate
directly (and separately) with the underlying RDBMS via standard SQL queries.
This allows any SQL-compatible RDBMS to be used fairly indiscriminately.

In addition to this, Grapht also includes a query processor, allowing simple
queries to be expressed in a syntax similar to SQL, but harnessing the
additional power of the graph-centric handler. Graph algorithms and queries
vary widely in requirements, such that it would be difficult to produce a single query language able to express the full spectrum of possible queries.
full diversity using a single query language. For this reason, the query
processor is only designed the be used  as a lightweight complement to the
Grapht API directly. 

(TODO: Diagram of main components)

%TODO: Diagram: (see paper somewhere on desk)

The Grapht API has a relatively small interface. Method invocations to the API
are for the most part handled exclusively by either the row-centric handler or the graph-centric
handler, allowing each to be highly optimised for the task at hand. A few
higher-level methods then allow client applications to combine the results of
graph and row-centric queries.  The \textit{row-centric handler} performs relatively
little work itself, relying instead on the decades of optimisations made in relational
databases to efficiently retrieve data by passing queries directly on to the
underlying RDBMS.

Within the \textit{graph handler}, data is laid out in memory to optimise graph-centric
access patterns. The handler contains an in-memory store which is by design of limited size so
that Grapht can be used on systems even with small amounts of memory
available. When a vertex is requested through the API which is not available
in the store, the vertex data is obtained from the relational database. At the
same time, a prefetcher is used to populate the store with adjacent vertices,
so that subsequent requests can hope to be fulfilled
directly from the store.

Finally, the \textit{query processor} receives string queries from a client application
and breaks them apart into row-centric subqueries targeting the relational
database directly, and graph-centric subqueries targeting the graph handler.
Client queries are written using an extension of SQL I call \textit{gSQL} (as
in \textit{graph}-SQL) inspired by Biskup et al's \textit{Graph Manipulation Kernel}\cite{gSQL}. The extension is a
strict superset of SQL, allowing pure-SQL subqueries to be passed on directly to
the underlying RDBMS. Pure-graph queries end up being handled exclusively by
the graph handler, and hybrid queries are broken apart and subqueries dispatched
appropriately.






\section{Row Handler}

As mentioned above, the row-centric handler performs relatively little work.
Instead, queries are passed unchanged to the underlying RDBMS, and the
resulting relations are passed back. One optimisiation is that queries to the
row-handler are  lazily evaluated by default. This allows queries to be
transformed or dropped altogether when doing so would be advantageous. One
situation in which this may arise is as part of a more complex API method
call. The method may need to join together results from a graph query and a
row-centric query. If the graph query returns an empty result set, the  row
query does not need to be evaluated -- the joined result will itself be empty -- so the row query may be dropped altogether.






\section{Graph Handler}
\label{impl:graph}

The graph-centric handler is slightly more involved than the row-handler. 
It contains an in-memory store, in which vertices are looked up before falling back to 
the underlying RDBMS. Internally, the store maintains an indexed list of vertices in
memory. Each of these vertices contains a list of edges, allowing for quick
traversal without needing to pass through an index again. If the full graph
could be loaded into memory, these edges would contain a pointer directly to the
target vertex, to avoid indirection through the vertex index.
In practice however, we do not expect to be able to always be able to contain
the entire graph in memory, and instead the edges contain a key to the vertex index, representing a vertex which may not yet be loaded into the store.

When a vertex is requested through the handler, the store's vertex index is first
checked to try to fulfill the request. If possible, the vertex is returned
directly. If this is not possible, the graph handler queries the underlying
RDBMS to obtain the vertex data. At the same time, a prefetcher is used
to obtain data for related edges and vertices. The requested vertex data is
immediately returned to the user and the prefetched related data
is added to a work queue. This work queue is processed in a separate
background process which is responsible for adding new items to the store, and
removing them when the size of the store exceeds the memory allowed by the current Java
virtual machine (JVM). When the size of the store does exceed the allowed
size, vertices are removed from the index according to a least-recently used
strategy. Each time a vertex is requested, the access is
logged, and the vertex with the
least recent access time is chosen when a vertex must be eliminated from the index. The idea behind this is that vertices
which have  been recently visited by the query processor are likely to be
visited again soon in the future, since the traversal of the graph will remain
local to the most recent vertices. Thus these vertices should be kept in the
store, in preference to vertices which were only visited a long time ago.

The performance of two different strategies (\textit{lookahead} and
\textit{block}) for prefetching was analysed in detail by Yoneki et
al.~\cite{crackle}, but I have here repeated and extended these results with
a third strategy. 

\subsection{Lookahead Prefetcher}

The first strategy examined by Yoneki et al.~simply performed a limited depth-first
traversal of the graph, starting from the requested vertex. This traversal was
performed using a recursive ``Common Table Expression'' (CTE). These types of
query allow a result set to be built up by recursively joining a partial
result set with another relation. As the lookahead depth increases, the size
of the partial result set grows exponentially, such that the cost of joining
becomes itself exponentially larger. For this reason, Yoneki identified that
after a certain depth, the cost of the prefetch procedure grows larger than
the avoided overhead. I replicated this finding, which is further discussed in
Chapter~\ref{ch:evaluation}.



\subsection{Block Prefetcher}

The second strategy used by Yoneki et al.~exploited the fact that there is
often a strong correlation between certain vertex attributes and graph locality. For
example, in a road network, the latitude and longitude of junctions can be
expected to correlate with graph locality, since nearby junctions are far more
likely to be connected by a road than distant ones. Yoneki terms this
``semantic'' locality, and uses it to quickly prefetch large blocks of the
graph by simply filtering on latitude and longitude. The cost of this
procedure does not grow as fast as for the lookahead prefetcher: for a block
of size $b$,  we would expect the cost of fetching all junctions within the
block to be proportional to the number of junctions within the block -- i.e.
the cost will be $O(b^2)$. Although this is advantageous from a performance
perspective, it places an important limitation on the data -- namely that
there must exist some attribute which is tightly correlated with graph
locality.



\subsection{Iterative Lookahead Prefetcher}

I examined a third possible strategy for this report, inspired by the first
two.  The lookahead prefetcher is most portable, since it can be
indiscriminately applied to any graph. However, it does not scale as well,
such that when the store is large, and it would be advantageous to prefetch
many vertices in one go, the performance cost is too high, and a block
prefetcher becomes more effective. The source of this cost penalty is that
many edges are unnecessarily traversed repeatedly. For example, after one
level of recursion, edges one hop away from the source are traversed, and
their destinations included within the CTE table. After five levels of
recursion, we should only be examining edges which are precisely five hops
away from the source. However the recursive CTE does not distinguish between
the edges which were added at the previous  level of recursion and those which
were added at the start of the procedure (and which do not need re-examination).

We can improve performance, then, by only expanding the fringes of the search
-- and even then only those fringes which have not been previously examined.
To achieve this, I implemented a slightly more involved prefetcher, which
maintains for each request a set of vertices explored this time. It then performs
a number of iterations -- one for each level of lookahead -- requesting just those 
vertices which were added to the fringe during the previous iteration. Requests
look like this: 

\begin{verbatim}
SELECT *
FROM edges
JOIN vertices
   ON vertices.id = edges.from_id
WHERE edges.from_id IN (1286, 1373, 1141, ...)
\end{verbatim}

These operations are very fast, since the \texttt{from\_id} field of the edge
relation is indexed such that retrieval is a constant time operation. This
type of indexing is not possible for the recursive CTE table. The trade-off
here is that a large number of requests are made to the RDBMS, such that the
communication and query-parsing latency are incurred repeatedly. However, as
is discussed in Chapter~\ref{ch:evaluation}, experiments showed that in
practice this latency was comparable the join cost of the CTE approach, for much
larger lookahead depths. In fact, this iterative lookahead approach allowed
lookaheads of almost ten times the optimal depth of the original lookahead
prefetcher for the same overall performance cost. 

Despite this, a difference was encountered in that the optimal depth for a
query was much more variable than for the CTE lookahead prefetcher. It is
likely that this optimality will vary based on factors such as store and graph
size as well as the degree of interconnectedness of the graph (since a graph
with many high-degree vertices will have a faster-growing fringe during
exploration). It was also found that the performance of the lookahead
prefetcher was higher than the block prefetcher, which contradicted earlier
results. As a result of these findings, a CTE lookahead prefetcher was used as
the default Grapht prefetcher with a depth of five hops.





\section{Grapht API}
\label{sec:grapht_api}

The Grapht API allows client applications to query the underlying RDBMS,
while  taking advantage of the row and graph-centric handlers to optimise
performance according to the expected access patterns. There is one API method
targeting the row-centric handler directly: \texttt{getRelation(sql)}. This
method takes a string parameter \texttt{sql} representing a standard SQL
query, and returns an iterable list of rows representing the result relation.
In order to target the graph-centric handler, one must first initialise the
graph store with a particular source. This is done by using the
\texttt{createGraph(vertices, vertexKey, edges, edgeKey, sourceKey, targetKey)} method, which
takes four string arguments. The arguments \texttt{vertices} and
\texttt{edges} correspond to SQL queries representing the edge and vertex
relations respectively. Normally, these strings will simply be the names of
the tables corresponding to edges and vertices in the underlying store, but it
is also possible to pass in a SQL query which will dynamically create an
edgelist. The remaining arguments, \texttt{vertexKey}, \texttt{edgeKey}, \texttt{sourceKey} and
\texttt{targetKey} are used to determine which fields of the chosen relations
correspond to vertex, edge, source and destination identifiers respectively.

Once the graph is initialised, two simple methods target the graph-centric
handler directly to retrieve the relevant entities from the graph according to
a given ID: \texttt{getVertex(id), getEdge(id)}. The objects returned by these
method calls have accessor methods of their own. Vertex objects have access to
the \texttt{getEdgeList} method to obtain a list of outgoing edges, and Edge
objects have access to a \texttt{getTargetVertex} method to obtain a reference
to the target vertex of this edge. As mentioned earlier, vertices are not
guaranteed to be present in the graph store when they are requested, as they
may have been displaced by the LRU policy. A request to an absent vertex will
trigger a fetch from the RDBMS. On the other hand, edges will
always be present in the store as long as their source vertex is. Finally, both Edge
and Vertex objects share a \texttt{getAttribute} method, used for retrieving
arbitrary data stored alongside the edge. This type of attribute accessor is common in graph databases where attributes are treated as having only secondary importance. This is tolerable here since queries where attributes are the priority can often pass through the row handler. Between these five methods, it is
possible to traverse the entire graph efficiently, starting and ending at any
arbitrary vertex, and collecting any desired attributes along the way.

Almost any algorithm can be efficiently expressed using this interface for
graph traversal and retrieval of attributes. For convenience, a few further methods
are provided, which use the methods described previously for themselves. The most
powerful of these is the \texttt{getPaths} method, which is used to discover
paths branching out from a source vertex. The design of this method is
closely tied to the design of the query processor (discussed in the next
section), since this method provides the bulk of the implementation for the
query processor. However, a client application can also call \texttt{getPaths}
independently.

(TODO: Diagram)

The operation of \texttt{getPaths} essentially follows a best-first
exploration of the graph starting from the source vertex. To achieve this, a
simple priority queue of partially-discovered paths is used. The priority of a
vertex is determined by a user-provided \textit{prioritiser} function. If no
prioritisation strategy is specified, an index is maintained such that all
elements inserted have a higher index than all those previously inserted. By
using this index as priority, the queue devolves into a last-in, first-out
stack as and search proceeds as a depth-first search. Although this is less
efficient than a native stack, since insertion becomes an $O(log n)$ operation
compared to $O(n)$, using a priority queue in this way offers more flexibility
to programmers.

Rather than provide a list of edges and vertices as output, a collection of
\textit{accumulators} are provided to \texttt{getPaths}. These are in effect
stateful functions, which  will receive new data at each step of traversal,
and update their internal state accordingly. For example, a \texttt{count}
accumulator could be provided, which simply increments its state with each new
vertex encountered. Alternatively, a \texttt{sum} accumulator could be
provided which will extract some attribute from the edges traversed, and add
the values together.

Along with these accumulators, \textit{evaluator} functions can be provided,
which determine whether the state of the accumulator is such that the current
path should be saved for inclusion in the result set. This allows us to, for
example, limit traversal to a depth of 3 by using a \texttt{count} accumulator
in conjunction with an evaluator which rejects paths where $\texttt{count}>3$.
An opportunity to cut the search space down is also provided to the user, who
can optionally specify  a rate of change for each accumulator, or failing this
at least a direction of change. For example, a \texttt{count} accumulator is
guaranteed to increase by one with each step through the graph, while a
\texttt{sum} accumulator wil at least increase each step if the attribute it
is accumulating is always positive. If an evaluator is provided which rejects
sums greater than 100, then we do not need to continue considering a paths'
possible expansions once it has already reached a sum of 100, since longer
paths will never have a smaller sum.

Two boolean values may also be provided, specifying that only unique edges or
vertices be used along the path. This is determined by maintaining a
``closed'' set of already-visited entities for each path. If a vertex or edge
which is already in a closed set is expanded, it is skipped rather than being
added to the queue as a new partial path.  Finally, a \textit{limit} parameter
is used to determine how many valid paths should be retrieved before
returning. By default, the limit is unset, and the search for valid paths will
be exhaustive.

To summarise, then, \texttt{getPaths} receives as mandatory input a graph and
start vertex, and the following optional parameters:

\begin{itemize}
    \item A \textit{prioritiser} function to guide traversal
    \item A list of vertex \textit{accumulators} (optionally with a rate of change specified)
    \item A list of edge \textit{accumulators} (optionally with a rate of change specified)
    \item A list of \textit{evaluators} to determine valid paths
    \item Two boolean values specifying whether vertices and edges must be unique along a path
    \item An integer \textit{limit} on the number of paths retrieved
\end{itemize}


In the special case where a single path is requested, an important
optimisation is possible. When multiple partial paths pass through the same
vertex, we need only consider the single route with the highest priority. In highly
interconnected graphs, this allows us to discard a large number of possible
paths early on. In practice by doing this, and requesting only unique
vertices, this  optimisation allows the best-first traversal to degenerate
into a variation on the A* search algorithm, as long as an admissible
heuristic is used in the prioritiser function to guide traversal.

On each iteration of the traversal, a partial path is fetched from the
priority queue. This path is considered to see if it satisfies the conditions
required to be output. If it does, the path is saved, and we check whether the
desired number of paths have been found, stopping execution if this is the
case. If not, we consider all possible extensions of this path by iterating
through the outgoing edges of the final vertex in the path. If it is possible
that these edges may lead to a valid solution (which can be determined by
examining the attributes' rates of change, as discussed earlier), then the
target vertices are fetched from the store. A new partial path is created,
accumulating attributes from both the edge and the target vertex. A priority
is calculated for this path according to the prioritiser function,  and the
new path is added to the queue, ready for selection in some future iteration.

The eventual output of \texttt{getPaths} is a list, where each entry contains
the accumulated results of traversing a valid path from the source vertex. In
effect, this result can be treated as a relational table, where each
accumulator is a column in the relation. This allows paths through the graph to
be manipulated just like any relation in the underlying RDBMS, and in
particular allows the results of a path-finding query to be joined with a
normal relation. This ability is a significant contribution of this work, since
obtaining this kind of result efficiently is not possible in either a purely
relational system (where  the path-finding component would take too long), nor
in a purely graph-oriented  system (where there exists no notion of a ``normal''
relation).  % TODO: explain better why graph solutions are not good enough

Ideally, the result relation of the \texttt{getPaths} method would be created
directly within the RDBMS, so that joins could literally be
performed as between normal relations. The current Grapht prototype is not sufficiently
tightly coupled enough to the PostgreSQL for this to be possible,
however. As a workaround in the meantime, the results of the \texttt{getPaths}
method can be loaded into the store after they are calculated as a temporary
in-memory relation. This does carry an inevitable performance  penalty
compared to being constructed directly in PostgreSQL's memory space.

It may seem as though only providing a path-searching method does not provide
a  very powerful abstraction beyond the vertex and edge-querying methods
provided.  However, the aggregation mechanism means that we are able to
consider attributes belonging to only vertices reachable from the source vertex.
For example,  we may wish to retrieve a list of all junctions reachable within
100km of some source junction. This can be achieved by using a
\texttt{sum} accumulator as described above and a \texttt{last} accumulator,
which simply retrieves attributes from the \textit{last} vertex along a path.
An evaluator can then be attached to the \texttt{sum} accumulator, checking
whether the total path distance is less than 100km. Since the sum is
non-decreasing, an exhaustive search will terminate as paths will cease to be
expanded after 100km. In this way, we can quickly retrieve a relation
containing just the details of reachable nearby junctions using a single API method call.






\section{Query Processor}

The aim of the query processor is to provide some of the power of the Grapht
API in a more lightweight manner, so that clients can simply express retrieval
queries as strings, rather than needing to interact with a more low-level API.
Biskup's Graph Manipulation Kernel\cite{gSQL} (GMK) unites relational and
graph models in a way similar to what we would like to achieve. It does this
by defining paths over graphs as relations with three implicit attributes:
\texttt{START}, \texttt{GOAL} and \texttt{LENGTH}.  By describing paths in
this way, we write queries over the possible search space of paths through
the graph, using a syntax very similar to SQL. For example, a ``friend-of-friend''-style query to find vertices two hops away from the vertex with
ID 1 would be written this way in Biskup's GMK:

\begin{verbatim}
SELECT *
    FROM PATHS OVER my_edge_relation
WHERE 
    START = 1 AND 
    LENGTH = 2 
\end{verbatim}

In this example, \texttt{my\_edge\_relation} is some relation with precisely two
attributes which are interpreted as start and end identifiers of edges in a
graph. If the relation contains more than two attributes, the attributes
indicating start and end may be included in parentheses, as in ``\texttt{...
PATHS OVER my\_edge\_relation (id1, id2)}.'' Other attributes of the edge
relation can then be aggregated along the path for inclusion in
the output path relation.  Although this provides a good starting
point for Grapht, a few notable shortcomings prevent us from adopting the
solution directly.

Firstly, Biskup's solution has no direct support for vertex attributes.
Vertices are in fact not treated as first-class entities at all, simply being
described implicitly as start and end attributes within the edge relation. It
is possible to include vertex attributes in a query by joining the edge
relation together with the vertex relation, however this results in
duplication of vertex data when several edges lead to the same vertex, and can
cause confusing ``off-by-one'' errors since any path will have one more vertex
than it has edges.

Secondly, Biskup's extension leaves the traversal mechanism used to find valid
paths unspecified. In many ways this is a good thing, and mimics the design of
SQL itself. By ensuring that the  query is fully declarative, an
implementation can select any method to find possible paths, and the client
application need not worry about efficiently finding solutions. In the absence
of cycles in the graph, the set of paths discovered will be constant, although
they may be discovered in a different order. This is no different to the way
in which the order of rows retrieved by a SQL query is undefined unless an
\texttt{ORDER BY} clause is present in the query. The difference is that the
performance gain from using an appropriate traversal order through the graph
is much more significant than the gains made by iterating through a set of
rows in a different order. If we want to find a particular neighbour of a
vertex, a depth-first search (DFS) might visit every single other vertex in the
graph before returning, while a breadth-first search (BFS) can terminate
almost immediately. 

Adding cycles to the graph makes appropriate selection of a traversal order
all the more critical, since DFS here may result in non-termination, while
BFS may have no problems. A final related problem with GMK is that it is
difficult to express uniqueness constraints on the vertices visited. By
enforcing that the vertices visited along a path be unique, it becomes
possible to traverse a graph with cycles without non-termination, so
expressing these constraints is very valuable.



\subsection{Improving on the Graph Manipulation Kernel}

To solve the issue of vertex attributes, two small changes must be made.
Firstly, defining a graph according to an edge relation is no longer
sufficient,  and graphs must now be specified as a pair consisting of an edge
relation and a vertex relation. As before, we must specify the attributes to
use as start and end points for the edge relation, and a vertex identifier for
the vertex relation. We must also now change the syntax of calculated
attributes of the output relation, to explicitly state whether values are
being accumulated along the edges of the path, or along the vertices. For
example, to accumulate the \texttt{name} attribute of vertices as a
concatenated string, along with the total \texttt{cost} of edges along the
path, a gSQL query may look like this:

\begin{verbatim}
SELECT START, END, LENGTH, 
    (ACC VERTICES CONCAT(name, " -> ")) path,
    (ACC EDGES SUM(0, cost)) cost 
FROM PATHS OVER (my_edge_relation(id1, id2), my_vertex_relation(id))
WHERE 
    START = 1
\end{verbatim}

In this way, the accumulation is made explicitly over vertices or edges
(\texttt{ACC VERTICES} in the former case, \texttt{ACC EDGES} in the latter).
Following the \texttt{ACC} keyword is the name of a function to use as
an accumulator (\texttt{CONCAT} or \texttt{SUM} in the example). For the
prototype version of Grapht, only a few functions were defined but others
would be trivial to add. The \texttt{CONCAT} function, for example, takes as
arguments the name of a attribute to accumulate at each step and a separator
string, returning the concatenated value of all named attributes along the
path.

To solve the remaining problems, I have added a \texttt{TRAVERSE} clause to
the end of the query format, which allows users to specify a prioritisation
order for traversing new vertices in search of a valid path. The clause is
optional, and if omitted a breadth-first search is used. An example use of this
clause would be:

\begin{verbatim}
SELECT * 
FROM PATHS OVER (my_edge_relation(id1, id2), my_vertex_relation(id))
WHERE 
    START = 1
TRAVERSE 
    UNIQUE VERTICES 
    BY -INDEX
\end{verbatim}

This example specifies a breadth-first traversal, avoiding cycles by
disallowing repeated vertices along the same path. The \texttt{UNIQUE}
modifier is again optional, and may specify either ``\texttt{VERTICES}'',
``\texttt{EDGES}'' or ``\texttt{VERTICES,EDGES}'' if duplicate edges and
vertices should both be avoided. ``\texttt{INDEX}'' refers to the insertion index within the internal priority queue discussed in Section~\ref{sec:grapht_api}. By ordering according to the negative index, we get a first-in, first-out queue, and hence follow a breadth-first search.

One limitation has been introduced to gSQL queries which was not present in
GMK: that queries should always specify a start vertex for the search.
This is included for performance reasons, following the implementation of the
\texttt{getPaths} method of the Grapht API.  In general it would be
inefficient to consider every possible path through the graph when the query
can be solved using a local traversal. It would be possible to reduce the
search space without using a known start vertex (for example by working
backwards from a known end vertex) but these strategies have not been
implemented in the initial prototype, such that a start vertex must always be
provided.




\subsection{Executing gSQL}

The main task of the query processor is to extract information from a gSQL
query in order to call the \texttt{getPaths} method of the API. When a gSQL
query is received, the first task is to separate components of the query into
row-centric and graph-centric queries. To achieve this, the query processor examines the data aource specified in the query. If data is being queried from one of the
relational tables, then that data is fetched through the row handler directly.
If data is being queried from a ``\texttt{PATHS OVER}'' clause, then the query
will instead be resolved through the graph handler. This provides a very
simple way to break apart a complex query and reliably identify what kind of
access pattern is expected for the query.

 When a query has been
identified as targeting the graph handler, it is further broken down into
actionable components. First, the data source is identified from the
\texttt{PATHS OVER} clause. If the precise combination of edge and vertex
relations has not been seen before, a new subgraph is initialised in the graph store
to represent it. Next, accumulator functions are generated to collect the data
required by the \texttt{SELECT} clause. Any accumulators specified in the
query are created according to the parameters passed in the query. In addition
to these, three accumulators are always implicitly created to represent the
\texttt{START}, \texttt{LENGTH} and \texttt{END} attributes of the output
relation. Finally, accumulators may also be  defined within the \texttt{WHERE}
clause, so that conditions can be checked at each step even if the result is
not included in the final relation. This \texttt{WHERE} condition is evaluated
at each vertex along a possible path.

The implicitly-created \texttt{LENGTH} accumulator is initialised with
a known rate-of change, since \texttt{LENGTH} will increase by one with each hop. This allows queries with a bound on \texttt{DEPTH} in the
\texttt{WHERE} clause to run exhaustively within this bound, since
Grapht knows that a shorter \texttt{LENGTH} can never arise further along the
path. Although this is a desirable behaviour, there is currently no way for
query writers to specify that other accumulated variables have a known rate of
change, although future versions of the query processor may provide
functionality for this.

The prioritiser and uniqueness constraints for \texttt{getPaths} are very
simply extracted from the \texttt{TRAVERSE BY} clause. Similarly, a limit  to
the number of paths to retrieve is read directly from the \texttt{LIMIT}
clause of the query.

gSQL can also describe complex clauses which join together results from both the row and graph handlers.
In these cases, a relational join can be performed as between normal relational
tables. This is possible since the result of a ``\texttt{PATHS OVER}'' query
(i.e. the output of a call to \texttt{getPaths}). As mentioned in Section~\ref{sec:grapht_api}, 
the Grapht prototype loads path relations as temporary tables with PostgreSQL, to allow the RDBMS to perform the join natively.

