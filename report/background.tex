\section{Relational Databases}

The relational database model first came to the fore in in the 1970s with
Codd's famous \textit{Relational Model of Data}~\cite{Codd}. The simplicity of this
model brought immediate success, and it quickly found favour in
industry at the time. Since then, the model has been carefully refined, with
numerous implementations each offering incremental optimisations.

Along with simplicity, most RDBMSs provide desirable guarantees such as
atomicity, consistency, isolation and durability (ACID). These properties were
formally defined by Gray~\cite{gray1981transaction}, and are now highly prized in a database system,
since they give a high degree of confidence in the integrity of data, and
allow database queries to be parallelised and distributed across several machines
without risk of conflict between multiple concurrent queries.

These factors have lead to widespread adoption of the relational database
model,  and a corresponding proliferation of implementations, including MySQL,
SQL Server and PostgreSQL. I will be evaluating PostgreSQL in this report, and
using it as a base RDBMS upon which to build a graph-centric hybrid
store. PostgreSQL is a popular, open-source database system supporting most of
the SQL standard. It also supports a procedural language called PL/pgSQL which
extends SQL to offer more programming control. In particular, PostgreSQL
supports recursive SQL queries which are crucial for efficiently finding
paths through an edge relation without performing exhaustive \texttt{JOIN}s.

\section{Graph Databases}

Although graph databases have enjoyed a recent resurgence in popularity, the
principles themselves are not much younger than Codd's relational database
model. Indeed, both models aim to represent relationships between entities with only slightly different focusses.
Chen~\cite{Chen} provided one of the earliest incarnations in the form of his
\textit{Entity-relationship Model.} This found popularity alongside the development
of object-oriented programming in the 1980s, and several variations to this
model were presented at the time~\cite{Kim}\cite{Abiteboul}.
Kunii~\cite{Kunii:1987:DGD:42040.42071} presented the first properly-called
graph database model, called \textit{G-Base}. Interest waned in the late 90s for a
variety of reasons, including a general move by the database research
community towards semistructured data. Recent times have seen a resurgence in
popularity, as alternative strategies are explored to handle the ``Big Data''
explosion in data size and complexity.

The need for graph representations of data can be seen through the number of
large technology companies who have recently developed in-house solutions to
the problem. In 2010, Facebook released Open Graph, and Twitter released
FlockDB\cite{pointer2010introducing}, both projects aiming to make traversal
through the social networks more efficient. In 2012 Google revealed their
Knowledge Graph~\cite{singhal2012introducing}, aiming to connect search terms
to real-world entities. The benefit of these graph databases is that queries
can be localised to a particular area within the graph. Without direct links
between related entities, every entity in the database would need to be
considered to determine whether it is related to another. This approach is
inefficient, particularly as the size of a database grows, and the cost of
traversing the  entire store increases.

It is important to note that graph database systems are different to graph
processing frameworks such as Ligra~\cite{shun2013ligra} or
Pregel~\cite{malewicz2010pregel}. Where a graph database system is concerned
with fast and frequent storage and retrieval of graph entities, a graph
processing system will typically perform more complex analysis and
transformations of the graph, often in a lower volume. The distinction is
similar to that between Online Transaction Processing (OLTP) and Online
Analytical Processing (OLAP) systems. An example query for a graph database
may be to find a route between two points in a graph, but an example query for
a graph processing system might be to calculate pagerank for the network. This
is an important distinction for our purposes, as the systems we discuss here
are concerned mainly with data retrieval. 

Some well-known graph databases include Titan, OrientDB and Neo4J. McColl et
al.~\cite{mccoll2014performance} provide a recent overview of these and
others, as well as a performance comparison. In this paper, I restrict my
attention to Neo4J. Neo4J is a schema-less database based on a key-value
store, which runs on the JVm and presents a Java API.

\section{Related Work}

The incumbency of relational databases coupled with the potential scalability
of  graph databases naturally prompts the question of whether it is possible
to efficiently store and query graph structured data in a relational database.
Indeed, several efforts have been made in this direction. 

Object-relational mapping (ORM) frameworks such as Hibernate ORM for Java
allow programmers to map an object-oriented domain model to a relational
database. This allows semantic links to be easily followed from one entity to
another by abstracting away from the underlying RDBMS. Although
ORM frameworks allow programmers to feel as though they are using a graph
database, they do not tend to focus on bringing any of the performance or
scalability advantages of graph databases to the relational world.

Another approach is taken by OQGraph\cite{oqgraph}, which provides a plugin
for  MariaDB and MySQL.  Users query a proxy
table, which OQGraph interprets as graph traversal instructions by imitating a storage engine. Although this
somewhat improves the programmer experience by providing a more graph-centric
interface, performance is not comparable to pure graph databases.
Additionally, since the plugin imitates a  storage engine, no parser
extensions are possible and queries must be expressed in  plain SQL. Although
this means that no new query language need be learnt, it also means that
graph-centric queries are awkward to express.

Biskup et al.~\cite{gSQL} described in 1990 a SQL extension which could be
used to  query graph relations. No implementation is provided, and again, no
performance considerations are made, but the work nonetheless provides a valuable 
inspiration for Grapht's query language \textit{gSQL}. Biskup's work is further discussed alonside gSQL in Chapter~\ref{ch:implementation}.

A more performance-oriented approach to bringing relational and graph
databases together can be found in Sun et
al.'s~\cite{Sun:2015:SER:2723372.2723732} SQLGraph. This approach uses
relational storage for adjacency information, and JSON storage for vertex and
edge attributes. Although this approach does improve performance for graph-centric queries, 
moving attribute information to JSON files hurts performance
of normal relational queries, which rely on fast and local access to attribute data to filter out rows.

Most recently, a Spark package has been developed called \textit{GraphFrames} which extends the
DataFrames library to handle graph data\cite{graphframes}.
DataFrames are a way of abstracting any underlying data source by loading it
into a Spark \textit{Resilient Distributed Dataset} (RDD). The GraphFrames
extension extends these RDDs by allowing them to be interpreted as graph adjacency
tables, and providing graph traversal methods. The query language provided  for GraphFrames is
somewhat limited, however, primarily designed for simple pattern-matching. It is not possible to guide the search direction through the graph, which is an ability that Grapht does have. Unlike Grapht, by building directly on top of Spark, GraphFrames queries
are easy to distribute and parallelise across a number of machines.

Finally, my work also takes inspiration from Yoneki et al.'s
\textit{Crackle}~\cite{crackle}, which  improves graph performance of a
relational store by maintaining a limited graph store in memory, which is
periodically filled using graph-aware prefetchers to ensure topologically
local vertices are preserved in the store. Crackle does not support pure-relational or hybrid queries, however,
nor does it attempt to present a declarative
query interface in the way that Grapht does. We replicate and extend some of
the Crackle results in Chapter~\ref{ch:evaluation}.
