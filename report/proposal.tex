\documentclass[11pt]{article}
\usepackage{a4wide,parskip,times}
\usepackage{color}
\usepackage{tabulary}

\usepackage[
backend=biber,
sorting=ynt
]{biblatex}
\addbibresource{main.bib}

\begin{document}

\begin{abstract}


There has recently been an increasing need for fast analysis of 
graph-structured data, which has led to the development of several graph-centric
alternatives to traditional relational databases. Although these ensure the
fast execution of queries which fit within this graph-centric model,
inevitably a compromise has been made, and other queries perform less
efficiently than they would have done with a relational database. I propose
the development of a query language enabling intelligent dispatch of optimised
queries either directly to a relational database, or through a graph-centric query processing
layer.

\end{abstract}

\section{Introduction, approach and outcomes}

Graph databases are an alternative to traditional relational databases, which
improve access speeds for some queries by prioritising access to related
entities, rather than expecting traversal through a fixed schema. This can
provide huge performance improvements for graph-centric queries, such as
shortest path queries. For other queries, however -- such as aggregation of
otherwise unrelated data items -- it is far more efficient to rely on a fixed
schema for fast access. A recent project\cite{crackle} has shown that it is
possible to unify both graph and relational database paradigms by providing
graph-focused pre-fetching of data from an underlying relational database.  I
now aim to further unify the systems in two ways.

My first contribution is to present a declarative query language 
targeting this hybrid system. Two interfaces to the data exist: going
straight to the relational database, or through the graph-centric query
layer. The language will be amenable to analysis, to
allow an interpreter to determine which queries will be 
handled more effectively by which interface, and to efficiently 
dispatch these targeted queries.

My second contribution is to extend the performance benefits
by decomposing
user queries and extracting opportunities for parallelisation
Decades of research have worked on
optimising SQL queries in this way, and some ideas from the domain of relational
query planning will provide insights into increasing the performance of graph
queries.

Another challenge of the work is ensuring usability of the query language. Queries targeting a graph database queries are a very different
to those for relational databases, as they are used to solve different sorts of problems. I will pay careful attention that both paradigms are sufficiently
expressible, and that the resulting language is coherent.

To evaluate my project, firstly, it will be important to show that the system provides a performance improvement
over current solutions. I will make comparisons against Neo4J\cite{neo4j} and PostgreSQL\cite{postgres}. I expect the two systems to run slightly faster on queries for which they are specialised, but that my system will significantly outperform them on queries for which they are not particularly optimised. I will also present an independent performance evaluation of my optimisations,
exploring their impact and limitations. Finally, I will evaluate the usability of the resulting
language to ensure that the platform is sufficiently expressive
without being incoherent.

% What about discussion of read-only accesses vs updates too?

If time allows, there are a few possible directions for extension. Firstly, 
the nature of the query layer places no restriction on
the type of the underlying data store. It would be interesting to explore the
effect of using other data sources in place of a relational database.
There may also be opportunities here for
query optimisation which are not present when using PostgreSQL. Finally, 
the discussion above has considered only the efficient retrieval of data.
Clearly it would also be desirable to allow updates, though this introduces
complexity with regards to the consistency of prefetched data. It would be
instructive to start work in this direction in order to have some idea of
the particular difficulties which would need overcoming.

\section{Work Plan}

The structure of my research will follow three main stages. Firstly, I will
work on the dispatch mechanism for user queries. I intend to identify optimal
dispatch strategies by taking a ``problem-first'' approach, and looking for
patterns in the differences between the kinds of problems which tend to
currently be approached using graph databases, and those which use relational
databases. I expect this work to roughly continue until the start of Lent
term.

During the course of Lent term, I will consider the problem of query
optimisation. I will proceed by performing tests to identify a few possible
avenues for performance improvements, before implementing the most promising
ones and measuring their effects. The design of the query language itself will
be ongoing throughout the entirety of the project, to ensure both that the
information retrieval problem space is adequately covered, and that
opportunities for optimisation are adequately exposed.

Finally, I will use the remaining time available to perform more detailed
evaluation, refine the usability of the language, and prepare the write up.

A detailed work plan is included overleaf. In this plan, bold items represent fairly
objective milestones, which should help ensure that I do not fall behind.


\begin{center}
\begin{tabulary}{\linewidth}{|CLLL|}
\hline
 Week & Date & Work to complete & \\
 \hline
 1  & 16th November--29th November & Background reading and prepare \textbf{revised proposal}. & \\[3ex]
 3  & 30th November--13th December & Set up development environment and \textbf{replicate Crackle results}. & \\[3ex]
 5  & 14th December--27th December & Expand Crackle results (different data/different queries). & \\[3ex]
 7  & 28th December--10th January & Collect results and \textbf{establish preliminary dispatch strategy}.  & \\[3ex]
 9  & 11th January--24th January & Read around area of query planning and optimisation. & \\[3ex]
 11 & 25th January--7th February & Perform tests to identify \textbf{two or three possible avenues for optimisation}. & \\[3ex]
 13 & 8th February--21st February & Implement first optimisations. & \\[3ex]
 15 & 22nd February--6th March & \textbf{Finish implementing optimisations} and \textbf{prepare progress report}.&\\[3ex]
 17 & 7th March--20th March & \textbf{Finish design of DSL}. & \\[3ex]
 19 & 21st March--3rd April & Conduct thorough performance evaluations. & \\[3ex]
 21 & 4th April--17th April & Evaluate DSL and begin write-up. & \\[3ex]
 23 & 18th April--1st May & Continue write-up. Complete an \textbf{initial draft of 11-13,500 words}.  & \\[3ex]
 25 & 2nd May--15th May & Refine draft and correct word count. & \\[3ex]
 27 & 16th May--29th May & Final edits and \textbf{submission}. & \\[3ex]
 (29)&30th May--6th June & Prepare \textbf{presentation}. & \\[3ex]
 \hline
\end{tabulary}
\end{center}

\end{document}
